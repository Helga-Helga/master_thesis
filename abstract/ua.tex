\chapter*{Реферат}

Дисертація містить \pageref{LastPage} сторінки,
\TotalValue{totalfigures} ілюстрацій і
\total{citenum} джерел літератури.

Задача бінокулярного стереобачення є
однією з актуальних проблем комп'ютерного бачення.
Вона лежить в основі побудови карти глибин
для отримання тривимірної моделі об'єкта чи поверхні.

Метою даної роботи є розробка методу прискорення алгоритмів стереобачення
за допомогою сегментації зображення.

Для досягнення мети було використано
\begin{itemize}
  \item відомості з геометрії в комп'ютерному бачення
        для правильної постановки задачі;
  \item відомості з теорії оптимізації та
        алгоритм дифузії для розв'язання оптимізаційної задачі;
  \item алгоритм викреслювання другого порядку для знаходження
        найкращої карти глибин після розв'язання задачі оптимізації.
\end{itemize}

\MakeUppercase{бінокулярне стереобачення,
               алгоритм дифузії,
               сегментація,
               карта глибин,
               тривимірна модель}
