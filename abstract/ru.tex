\chapter*{Реферат}

Диссертация содержит \pageref{LastPage} страниц,
\TotalValue{totalfigures} иллюстраций и список использованной литературы из
\total{citenum} наименований.

Задача бинокулярного стереозрения
является одной из актуальных проблем компьютерного зрения.
Она лежит в основе построения карты глубин для получения трёхмерной
модели объекта или поверхности.

Цель данной работы~---~разработка метода ускорения
алгоритмов стреозрения с использованием сегментации изображения.

Для достижения цели были использованы
\begin{itemize}
  \item сведения из геометрии
        в компьютерном зрении для правильной постановки задачи;
  \item сведения из теории оптимизации и
        алгоритм диффузии для решения оптимизационной задачи;
  \item алгоритм вычёркивания второго порядка для нахождения наилучшей
  карты глубин после решения задачи оптимизации.
\end{itemize}

\MakeUppercase{бинокулярное стереозрение,
               алгоритм диффузии,
               сегментация,
               карта глубин,
               трёхмерная модель}
