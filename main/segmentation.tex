\chapter{Сегментація зображення для прискорення алгоритмів стереобачення}

У третьому розділі запропоновано спосіб прискорення алгоритмів стереобачення
на прикладі алгоритму дифузії за допомогою сегментації зображення.

\section{Сегментація зображення}

Ліве зображення $L$ з вхідної стереопари
розбивається на прямокутну решітку з однаковим заданим розміром комірок.
Всі пікселі, що належать одній комірці,
діляться на дві групи за середньою інтенсивністю пікселів комірки
(рис.~\ref{fig:superpixels:visualization}).
До першої групи (зображена білим на рисунку) відносять пікселі,
інтенсивність яких не перевищує середньої інтенсивності пікселів комірки.
Піксель з координатами $\left(x^*, y^* \right)$, що належить даній комірці $S$
потрапляє до першої групи, якщо
\begin{equation*}
    L \left(x^*, y^* \right) \le
        \frac{1}{ \left| S \right| }
        \sum \limits_{\left(x, y \right) \in S} L \left(x, y \right),
\end{equation*}
де $ \left| S \right|$~---~кількість пікселів у комірці.
До другої групи, що зображена чорним на рисунку,
відносять всі інші пікселі комірки, що не потрапили в першу групу.
Кожну таку групу будемо називати \textit{суперпікселем}.
Таким чином, в процесі сегментації ліве зображення
стереопари розбивається на комірки, в яких знаходиться два суперпікселя:
більш світлий і більш темний.

\begin{figure}[h]
\centering
    \begin{subfigure}[t]{0.4\textwidth}
        \centering
        \includegraphics[width=\textwidth]{images/cell}
        \caption{Комірка зображення}
        \label{fig:cell}
    \end{subfigure}
    \hfill
    \begin{subfigure}[t]{0.4\textwidth}
        \centering
        \includegraphics[width=\textwidth]{images/superpixels}
        \caption{Комірка зображення, розбита на два суперпікселя}
        \label{fig:superpixels}
    \end{subfigure}
    \caption{Візуалізація суперпікселів}
    \label{fig:superpixels:visualization}
\end{figure}

Після запропонованої сегментації будується
$\left| T_s \right|$-дольний граф з множиною координат суперпікселів
\begin{equation*}
    T_s = \left\{
        \left(x_s, y_s, i \right) \; \middle| \;
        1 \le x_s \le m, \,
        1 \le y_s \le n, \,
        i \in \left\{ 0, 1 \right\}
    \right\},
\end{equation*}
де $m$~---~кількість комірок по горизонталі,
$n$~---~кількість комірок по вертикалі.
Тепер кожній долі графу відповідає один суперпіксель~---~об'єкт графу.
Кожен об'єкт $\left(x_s, y_s, i \right) \in T_s$ містить
$ \left| D \right|$ міток (або вершин),
які відповідають всім можливим зсувам горизонтальної координати пікселів,
що належать об'єкту.
Таким чином, всі пікселі, що належать одному суперпікселю,
будуть мати однакову глибину (або паралакс) на результуючій карті глибин,
тобто тепер горизонтальний зсув $d \in D$
буде шукатись не для кожного пікселя окремо, а для цілої групи пікселів,
які утворюють суперпіксель, одразу.

В об'єкті $\left(x_s, y_s, i \right) \in T_s$ на вершину з міткою $d \in D$
накладається штраф,
який дорівнює сумі штрафів за вибір відповідних вершин у всіх пікселях,
які належать даному суперпікселю,
\begin{equation*}
    f_{\left(x_s, y_s, i\right)}^s \left( d \right) =
    \sum \limits_{\left(x, y \right) \in \left(x_s, y_s, i \right)}
        f_{\left(x, y \right)} \left( d \right),
\end{equation*}
де $f_{\left(x, y \right)} \left( d \right)$~---~штраф за вибір вершини,
введений в першому розділі в формулі \eqref{penalty:vertex}.

Кожен об'єкт може мати до дев'яти сусідів: по два об'єкти в верхній, правій,
нижній і лівій комірках, а також другий об'єкт,
який належить тій же комірці (рис.~\ref{fig:superpixel:neighbors}).
Об'єкти, що відповідають коміркам на краях зображення, мають по сім сусідів,
а об'єкти, що відповідають кутовим коміркам,~---~по п'ять.
Штраф, який накладається на дужки між вершинами різних об'єктів, такий же,
як і при постановці задачі без суперпікселів, тобто дорівнює
$g_{\left(x_s, y_s, i \right), \left(x_s', y_s', i' \right)}
    \left(d, d' \right)$,
де $\left(x_s, y_s, i \right) \in T_s$ та
$\left(x_s', y_s', i' \right) \in T_s$~---~сусідні об'єкти,
$d$~---~мітка в об'єкті $\left(x_s, y_s, i \right)$,
а $d'$~---~мітка в об'єкті $\left(x_s', y_s', i' \right)$.
Множину всіх пар сусідніх об'єктів в цьому графі позначимо через $\mathcal{N}^s$.

\begin{figure}[h]
  \centering
  \includegraphics[width=0.7\textwidth]{images/neighbours_superpixel}
  \caption{Структура сусідства при використанні суперпікселів.
           Квадратами позначені комірки,
           що містять по два суперпікселя (об'єкта):
           світлий ($i = 0$) і темний ($i = 1$).
           Стрілками позначені дуги,
           що виходять зі світлого суперпікселя центральної комірки в об'єкти,
           що є для нього сусідніми}
  \label{fig:superpixel:neighbors}
\end{figure}

Аналогічно штрафній функції \eqref{eq:overview:penalty}
вихідної задачі отримуємо штрафну функцію модифікованої задачі
\begin{equation*}
\begin{gathered}
    G_s \left( \pmb{d} \right)
    = \sum \limits_{x = 1}^{m}
        \sum \limits_{y = 1}^{n}
            \sum \limits_{i \in \left\{ 0, 1 \right\}}
                f_{\left( x_s, y_s, i \right)}^s
                    \left( d \left(x_s, y_s, i \right) \right) + \\
    + \sum \limits_{\left( \left(x_s, y_s, i \right), \left(x_s', y_s', i' \right) \right) \in \mathcal{N}^s}
            g_{\left(x_s, y_s, i \right), \left(x_s', y_s', i' \right)} \left(
                d \left( x_s, y_s, i \right), d \left( x_s', y_s', i' \right)
            \right).
\end{gathered}
\end{equation*}

Далі задача розв'язується методом, що описаний у попередньому розділі.

\section{Складність алгоритму дифузії}

% TODO: diffusion complexity

\section*{Висновки до розділу 3}
\addcontentsline{toc}{section}{Висновки до розділу 3}

Запропоновано спосіб прискорення розв'язання задачі стереобачення
за допомогою алгоритма дифузії.
Прискорення алгоритму базується на зменшенні розміру графу,
на якому розв'язується задача, за допомогою сегментації зображення так,
щоб не втратити багато інформації та
не сильно погіршити результуючу карту глибин.
