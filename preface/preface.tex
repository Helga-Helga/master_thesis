\chapter*{Вступ}
\addcontentsline{toc}{chapter}{Вступ}

Значною мірою робота завдячує Євгену Валерійовичу Водолазському~---~старшому
науковому співробітнику Відділу обробки та розпізнавання образів
Міжнародного науково-навчального центру інформаційних технологій
і систем НАН України та МОН України.

\textbf{Актуальність роботи.}
За рахунок зростання актуальності автоматичної обробки
та аналізу візуальної інформації розробляється багато методів
розв'язання задач комп'ютерного бачення, в тому числі стереобачення.
Остання задача виникає при проектуванні інтелектуальних робототехнічних
комплексів, систем управління рухомими апаратами,
при проведенні біомедичних досліджень.
Задача стереобачення полягає в побудові карти глибин за двома зображеннями,
що лежить в основі отримання тривимірної моделі об'єктів і поверхонь.
Алгоритм дифузії~---~один з популярних методів для розв'язання задачі
стереобачення, який часто показує гарні практичні результати,
але працює досить повільно.
Таким чином, виникає необхідність у розробці більш швидких методів,
які дають не гірші результати.

\textbf{Мета і завдання дослідження.}

\textit{Об'єкт дослідження}~---~побудова карти глибин за двома зображеннями.

\textit{Предмет дослідження}~---~алгоритми стереобачення.

Метою роботи є розробка методу прискорення алгоритмів стереобачення
з використанням сегментації зображення.

Завдання наступні:
\begin{enumerate}
  \item
    ознайомитися з задачею стереобачення та існуючими методами її розв'язання;
  \item
    ознайомитися з алгоритмом дифузії,
    що використовується для розв'язання задачі оптимізації,
    що виникає при розв'язанні задачі;
  \item
    ознайомитися з алгоритмом викреслювання другого порядку,
    що використовується для знаходження однієї з найкращих карт глибин
    після розв'язання оптимізаційної задачі;
  \item
    запропонувати та перевірити новий метод прискорення
    алгоритмів розв'язання задачі стереобачення на прикладі алгоритму дифузії;
  \item
    розробити програмну реалізацію алгоритму та його модифікації.
\end{enumerate}

\textbf{Методи дослідження:}
\begin{enumerate}
    \item збір інформації, опрацювання літератури за темою;
    \item теоретична та практична перевірка роботи алгоритму стереобачення;
    \item аналіз отриманих результатів.
\end{enumerate}

\textbf{Наукова новизна одержаних результатів.}

В роботі чітко описана постановка задачі стереобачення
та її розв'язання за допомогою алгоритму дифузії.
Запропоновано новий метод зменшення складності даного алгоритму.

\textbf{Практичне значення одержаних результатів.}

Карти глибин, побудовані за допомогою алгоритмів стереобачення,
можна використовувати для відновлення поверхонь за допомогою двох зображень.
Був описаний та перевірений
новий спосіб прискорення алгоритмів за допомогою сегментації зображення,
за якого не втрачається багато інформації про глибину об'єктів.

\textbf{Публікації.}

XVIII Всеукраїнська науково-практична конференція студентів,
аспірантів та молодих вчених <<Теоретичні і прикладні проблеми фізики,
математики та інформатики>>.
