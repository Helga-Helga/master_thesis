\chapter*{Перелік умовних позначень, символів, одиниць, скорочень і термінів}
\addcontentsline{toc}{chapter}{Перелік умовних позначень, символів, одиниць, скорочень і термінів}

\section*{Стандартні позначення}
\renewcommand{\arraystretch}{0.95}
\noindent\begin{tabular}{ @{}l p{14.7cm} }
 $\mathbb{N}$                       & множина натуральних чисел \\
 $\mathbb{N}_+$                     & множина додатних натуральних чисел \\
 $\mathbb{R}$                       & множина дійсних чисел \\
 $\mathbb{R}^n$                     & $n$-вимірний векторний простір над полем дійсних чисел \\
 $\mathbb{R}_+^n$                   & множина векторів з невід'ємними координатами в $\mathbb{R}^n$ \\
 $\Delta^n$                         & $n$-вимірний симплекс \\
 $\Delta^X$                         & $\left| X \right|$-вимірний симплекс, координати якого проіндексовані елементами з множини $X$ \\
 $\left| X \right| $                & потужність множини $X$, або кардинальне число множини $X$ \\
 $conv \left( X \right)$            & опукла оболонка множини $X$ \\
 $X \times Y$                       & декартів добуток множин $X$ та $Y$ \\
 $\langle \pmb{x}, \pmb{y} \rangle$ & скалярний добуток векторів $\pmb{x}$ та $\pmb{y}$ \\
 $\mathcal{O} \left( \cdot \right)$ & <<О>> велике, нотація Ландау
\end{tabular}

\section*{Позначення, введені в дисертації}
\renewcommand{\arraystretch}{0.95}
\noindent\begin{tabular}{ @{}l p{14.7cm} }
 $T$                            & множина координат пікселів зображення \\
 $C$                            & множина інтенсивностей пікселів зображення \\
 $I \left( x, y \right)$        & інтенсивність пікселя з горизонтальною координатою $x$ і вертикальною координатою $y$ на зображенні $I$ \\
 $\mathcal{N} \left( v \right)$ & множина вершин графу, інцидентних вершині $v$ \\
 $\mathcal{N}$                  & множина пар інцидентних вершин графу
\end{tabular}
